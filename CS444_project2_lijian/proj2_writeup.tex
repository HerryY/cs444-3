\documentclass[letterpaper,11pt,titlepage]{article}

\usepackage{graphicx}                                        

\usepackage{amssymb}                                         
\usepackage{amsmath}                                         
\usepackage{amsthm}                                          

\usepackage{alltt}                                           
\usepackage{float}
\usepackage{color}

\usepackage{url}

\usepackage{balance}
\usepackage[TABBOTCAP, tight]{subfigure}
\usepackage{enumitem}

\usepackage{pstricks, pst-node}

\usepackage{geometry}
\geometry{textheight=10in, textwidth=7.5in}

\usepackage{listings}
\usepackage{multirow, booktabs, array}

\definecolor{mygreen}{rgb}{0,0.6,0}
\definecolor{mygray}{rgb}{0.5,0.5,0.5}
\definecolor{mymauve}{rgb}{0.58,0,0.82}

\lstset{ %
  basicstyle=\ttfamily,            % the size of the fonts that are used for the code
  breakatwhitespace=false,         % sets if automatic breaks should only happen at whitespace
  breaklines=true,                 % sets automatic line breaking
  captionpos=b,                    % sets the caption-position to bottom
  commentstyle=\color{mygreen},    % comment style
  escapeinside={\%*}{*)},          % if you want to add LaTeX within your code
  extendedchars=true,              % lets you use non-ASCII characters; for 8-bits encodings only, does not work with UTF-8
  keepspaces=true,                 % keeps spaces in text, useful for keeping indentation of code (possibly needs columns=flexible)
  keywordstyle=\color{blue},       % keyword style
  numbers=left,                    % where to put the line-numbers; possible values are (none, left, right)
  numbersep=10pt,                  % how far the line-numbers are from the code
  numberstyle=\tiny\color{mygray}, % the style that is used for the line-numbers
  rulecolor=\color{black},         % if not set, the frame-color may be changed on line-breaks within not-black text (e.g. comments (green here))
  showspaces=false,                % show spaces everywhere adding particular underscores; it overrides 'showstringspaces'
  showstringspaces=false,          % underline spaces within strings only
  showtabs=false,                  % show tabs within strings adding particular underscores
  stepnumber=1,                    % the step between two line-numbers. If it's 1, each line will be numbered
  stringstyle=\color{mymauve},     % string literal style
  tabsize=8,                       % sets default tabsize to 8 spaces
  %title=\lstname                  % show the filename of files included with \lstinputlisting; also try caption instead of title
}

\lstdefinestyle{niceC}{
  belowcaptionskip=1\baselineskip,
  breaklines=true,
  frame=L,
  xleftmargin=0.2in,
  language=C,
  showstringspaces=false,
  basicstyle=\footnotesize\ttfamily,
  keywordstyle=\bfseries\color{green!40!black},
  commentstyle=\itshape\color{purple!40!black},
  identifierstyle=\color{blue},
  stringstyle=\color{orange},
}

%random comment

\newcommand{\cred}[1]{{\color{red}#1}}
\newcommand{\cblue}[1]{{\color{blue}#1}}

\usepackage{hyperref}

\def\name{Jianzhi Li}

%% The following metadata will show up in the PDF properties
\hypersetup{
  colorlinks = true,
  urlcolor = black,
  pdfauthor = {\name},
  pdfkeywords = {cs444 ``operating systems II'' Project 2},
  pdftitle = {CS 444 Project 2: Write-up},
  pdfsubject = {CS 444 Project 2},
  pdfpagemode = UseNone
}

\parindent = 0.0 in
\parskip = 0.2 in

\usepackage{sectsty}


\sectionfont{\fontsize{12}{15}\selectfont}


\begin{document}


Jianzhi Li\newline
CS444 Project 2 Write-up

\section{Design for implement the SSTF algorithms}

For my design, first of all, most of functions will modified based on the 'noop-iosched.c'. Since the SSTF is always looking for the 
shortest distance between current head and the nearest service request. So I need design two important features for SSTF.

The first feature is sort all nodes and always serve from the nearest request to the furthest request. I modified the 'noop add request'
function that allowed me to sort nodes when each time a request add to the request queue. For the first service request since there is no other requests compare with
it, so what ever the distance is it I do need add that first request in the queue. And I do need know the location of the sector and the location of the service request that I could 
know which request is the nearest one. For doing that, I need list each of node in the queue and located each node that can compare those nodes then sort them.
To know the distance between the location of the sector and the request. The second feature is to merge the requests. According to the result of the sort feature, if everything works correctly,
I already have a queue have been sorted as what SSTF algorithm suppose to be. For example, there has request 1 and request 2. If request 2 is more nearly than the request 1 then I need
merge request 2 in front of request 1. And there should have the last one that is never be added in the queue that is the furthest one since that is never won a comparison with other requests.   



\section{Verison Control Log}

\begin{table}[!hbp]
\begin{tabular}{|c|}

\hline
commit 9df5543d850b5ebf9e82d74b6d8d3cff2b61e3ee\\
Author: lijian $<lijian@os-class.engr.oregonstate.edu>$\\
Date:   Mon Oct 26 22:25:50 2015 -0700\\
finised sstf and the patch file\\

\hline
commit 56bda3a646c2256dbe15e25bcbab15c4dca1106e\\
Author: lijian $<lijian@os-class.engr.oregonstate.edu>$\\
Date:   Sun Oct 25 19:34:33 2015 -0700\\
update wirte-up and tex Makefile\\

\hline
commit 444082120b6d808f4d4554e911b8da70aa3a728b\\
Author: lijian $<lijian@os-class.engr.oregonstate.edu>$\\
Date:   Thu Oct 22 19:28:47 2015 -0700\\
change folder name\\

\hline
commit af182c887b3e686086e2397c693c44abbb81577f\\
Author: lijian $<lijian@os-class.engr.oregonstate.edu>$\\
Date:   Thu Oct 22 19:25:25 2015 -0700\\
con2 finished\\


\hline
commit cbe43c546bf312ec1ba0831c187c3d02848984e3\\
Author: lijian $<lijian@os-class.engr.oregonstate.edu>$\\
Date:   Mon Oct 19 20:39:18 2015 -0700\\
project 2 created\\


\hline
commit ccd3d15153e9973c14c01faa0dcdb87175764ed8\\
Author: lijian $<lijian@os-class.engr.oregonstate.edu>$\\
Date:   Mon Oct 19 12:41:44 2015 -0700\\
start working with con 2\\

\hline

 
\end{tabular}
\end{table}

\section{Work Log}
\begin{enumerate}

\item Monday, 10/19/15. Start working on the project2. Catch up for readings and research on SSTF.

\item Tuesday, 10/20/15. Modified the code from "noop-iosched.c" to "sstf-iosched.c", simply replace every noop to sstf. 
Have not figure out how those I/O scheduler can run on kernel. And trying to disable virtio on VM. Did some research that I can change the
disk type before I boot the VM but every time I changed the type of disk the VM cannot boot and stuck on a process. 

\item Thursday, 10/22/15. Still have no clued how to disable the virtio either inside or outside the VM. The implementing of sstf is going on. 
Trying to use sorting algorithms from previously study such as bubble sort and merge sort but that seems like I cannot transplant those algorithm
in my source file since that is in kernel level. And since this is the first time I programming on kernel level I found I cannot compile it by using gcc and 
have no idea how to check that correctness.  

\item Friday, 10/23/15. Figured out how to disable the virtio. I removed "if" statements when I boot the VM also changed the vda to the hda. 

\item Sunday, 10/25/15. Done most of what I designed for SSTF but not sure how to test those code on kernel. 

\item Monday, 10/26/15. Be honest, I still do not know how to test my source file on kernel. But I modified "Kconfig.iosched" and "Makefile" in the block of the kernel. And 
I can selected SSTF IO scheduler by using "make menuconfig". Compare the "Makefile", "Kconfig.iosched" and the source file of SSTF to the patch file. Otherwise, 
everything finished with this assignment expect well test the correctness with my source file. 

\end{enumerate}



\section{Questions}
\begin{enumerate}

\item I think the main point of this assignment is to deeply understand the principle of the IO scheduler, also the different between each IO scheduler. Otherwise, this assignment made me can 
play with kernel deeply such as to modified kernel files. Also to made the patch file to record what changed between the previously version of file and the new version of file. That is a important
skills to be a kernel programmer.  

\item For this assignment, I have a template which is "noop-iosched.c". The NOOP IO scheduler that is a simply FIFO that is do less job for IO scheduler so that is a good template to use for implement.
I do not have deep understanding for SSTF algorithm. I just know how that works and used simple sort algorithm to try to make that work. Since I do not have good test skills on this assignment, I cannot 
say my algorithm exactly work as same as SSTF. 

\item As previously I mentioned. I tried to test my SSTF source file to see if that work. But after I did some research I still do not know how to test my source file on kernel. But I do successfully add
my SSTF scheduler in the kernel and that can be selected.  

\item In this assignment, I learned how to disable the virtio then boot the VM. And how to create the patch file. Also the concepts of IO schedulers and did some modified for kernel files. If I can run my source file
correctly in the kernel I could see more how IO scheduler works in the kernel that I will learn more in this assignment.

\end{enumerate} 




\end{document}