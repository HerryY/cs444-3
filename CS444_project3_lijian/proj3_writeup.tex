\documentclass[letterpaper,11pt,titlepage]{article}

\usepackage{graphicx}                                        

\usepackage{amssymb}                                         
\usepackage{amsmath}                                         
\usepackage{amsthm}                                          

\usepackage{alltt}                                           
\usepackage{float}
\usepackage{color}

\usepackage{url}

\usepackage{balance}
\usepackage[TABBOTCAP, tight]{subfigure}
\usepackage{enumitem}

\usepackage{pstricks, pst-node}

\usepackage{geometry}
\geometry{textheight=10in, textwidth=7.5in}

\usepackage{listings}
\usepackage{multirow, booktabs, array}

\definecolor{mygreen}{rgb}{0,0.6,0}
\definecolor{mygray}{rgb}{0.5,0.5,0.5}
\definecolor{mymauve}{rgb}{0.58,0,0.82}

\lstset{ %
  basicstyle=\ttfamily,            % the size of the fonts that are used for the code
  breakatwhitespace=false,         % sets if automatic breaks should only happen at whitespace
  breaklines=true,                 % sets automatic line breaking
  captionpos=b,                    % sets the caption-position to bottom
  commentstyle=\color{mygreen},    % comment style
  escapeinside={\%*}{*)},          % if you want to add LaTeX within your code
  extendedchars=true,              % lets you use non-ASCII characters; for 8-bits encodings only, does not work with UTF-8
  keepspaces=true,                 % keeps spaces in text, useful for keeping indentation of code (possibly needs columns=flexible)
  keywordstyle=\color{blue},       % keyword style
  numbers=left,                    % where to put the line-numbers; possible values are (none, left, right)
  numbersep=10pt,                  % how far the line-numbers are from the code
  numberstyle=\tiny\color{mygray}, % the style that is used for the line-numbers
  rulecolor=\color{black},         % if not set, the frame-color may be changed on line-breaks within not-black text (e.g. comments (green here))
  showspaces=false,                % show spaces everywhere adding particular underscores; it overrides 'showstringspaces'
  showstringspaces=false,          % underline spaces within strings only
  showtabs=false,                  % show tabs within strings adding particular underscores
  stepnumber=1,                    % the step between two line-numbers. If it's 1, each line will be numbered
  stringstyle=\color{mymauve},     % string literal style
  tabsize=8,                       % sets default tabsize to 8 spaces
  %title=\lstname                  % show the filename of files included with \lstinputlisting; also try caption instead of title
}

\lstdefinestyle{niceC}{
  belowcaptionskip=1\baselineskip,
  breaklines=true,
  frame=L,
  xleftmargin=0.2in,
  language=C,
  showstringspaces=false,
  basicstyle=\footnotesize\ttfamily,
  keywordstyle=\bfseries\color{green!40!black},
  commentstyle=\itshape\color{purple!40!black},
  identifierstyle=\color{blue},
  stringstyle=\color{orange},
}

%random comment

\newcommand{\cred}[1]{{\color{red}#1}}
\newcommand{\cblue}[1]{{\color{blue}#1}}

\usepackage{hyperref}

\def\name{Jianzhi Li}

%% The following metadata will show up in the PDF properties
\hypersetup{
  colorlinks = true,
  urlcolor = black,
  pdfauthor = {\name},
  pdfkeywords = {cs444 ``operating systems II'' Project 3},
  pdftitle = {CS 444 Project 3: Write-up},
  pdfsubject = {CS 444 Project 3},
  pdfpagemode = UseNone
}

\parindent = 0.0 in
\parskip = 0.2 in

\usepackage{sectsty}


\sectionfont{\fontsize{12}{15}\selectfont}


\begin{document}


Jianzhi Li\newline
CS444 Project 3 Write-up

\section{Design Plan For Ram Disk with Crypto API}

Since the crypto API is only for encryption when writing data in ram disk and decrypt when read data from ram disk.
So I need separate my work for two parts. The first part I need have a ram disk driver. For that part, there has some hint
in the assignment requirement that is in LDD3 which is 'Linux Device Drivers, Third Edition' that has a simple that has the
same function as a block device driver. I would use that simple from a source file called 'sbd.c' and just use crypto API to make that device driver can have 
encrypts and decrypts data  when that write and read data. The data before encrypts and after encrypts that is would be different, also
the data before decrypts and after decrypts when the disk wants to read. After some research, the crypto API is contains many kinds of crypto
algorithms, but for the ram disk that is not keep data for a long time, data just temporary be stored in the ram disk. So I just need do encrypts and 
decrypts when the ram risk written or read data that is not necessary to use complicated algorithm but just filter data. All I need to do is to modified 'sbd transfer'  function which is to write and read data. 
    




\section{Work Log}
\begin{enumerate}

\item Monday, 11/02/15. Created project3 on git and start research ram disk. Reread Chapter 17.

\item Tuesday, 11/03/15. Start looking at LDD3 and trying to find a simple block driver.
\item Thursday, 11/05/15. Copied 'sbd.c' as my block device driver. Start going through each function.

\item Friday, 11/06/15. Located the function 'sbd transfer', that is the function I need add crypto API since there is the encrypts and decrypts function will operate the data. 
\item Sunday, 11/08/15. Some syntax problem with crypto API. There is too many of them in the introduction of crypto API. Most of them are struct. For doing that, I may just need take out encrypts and decrypts function from struct and put them in 'sbd transfer' function.

\item Monday, 11/09/15. Finished the crypto API, I have print function under each step of encryption and decryption which mean before and after. I tried to rebuild the kernel and fixed some bugs then that is working well. But I have a concern that is I do not know why the Makefile cannot give me the kernel object file. That is a problem to change to mount the ram disk driver. 
I tried to use 'insmod' to install the module but there was an error that is my object file have no permission to do that. 

\end{enumerate}


\section{Questions}
\begin{enumerate}

\item I think the main point for this assignment to let me learn how to use module in the linux kernel. Since if I want to be a kernel programmer, I have to learn module programming since that is an important part of the kernel. That is also a way how kernel programmer to improve kernel and add more functions and features in the kernel.
Also module as my understanding that is support hot plug which mean I can modify a module when the VM is running. I think that is also how the USB working in the operating system.

\item Be honest, the 'sbd.c' gives most of what I need in this assignment, all I need is to do is to figure out how to use crypto API. I can rebuild my kernel that is mean I do not have bugs in my code but that is not mean I do not have problem with the function of the code.

\item As I metioned in my work log. I do want to test my code that add that to be a module. But once I want to install my module that I would receive 'operation not permitted'. I would fix that as fast I can.

\item In this assignment, I learn how to use crypto API and concept of module and block devices as what Chapter 17 in our textbook that addressed. For install a module to a running VM and change the file system those are also important technique to learn. As I am not successful to test my correctness before the due date. I would figure that out before the demo. 

\end{enumerate} 


\section{Verison Control Log}

\begin{table}[!hbp]
\begin{tabular}{|c|}

\hline
commit 5eb0afbde21742c0d4815f265ecf8dfcb6642c22\\
Author: lijian $<lijian@os-class.engr.oregonstate.edu>$\\
Date:   Mon Oct 26 23:44:13 2015 -0700\\
writeup finised\\

\hline
commit 9df5543d850b5ebf9e82d74b6d8d3cff2b61e3ee\\
Author: lijian $<lijian@os-class.engr.oregonstate.edu>$\\
Date:   Mon Oct 26 22:25:50 2015 -0700\\
finised sstf and the patch file\\

\hline
commit 56bda3a646c2256dbe15e25bcbab15c4dca1106e\\
Author: lijian $<lijian@os-class.engr.oregonstate.edu>$\\
Date:   Sun Oct 25 19:34:33 2015 -0700\\
update wirte-up and tex Makefile\\

\hline
commit 444082120b6d808f4d4554e911b8da70aa3a728b\\
Author: lijian $<lijian@os-class.engr.oregonstate.edu>$\\
Date:   Thu Oct 22 19:28:47 2015 -0700\\
change folder name\\

\hline
commit af182c887b3e686086e2397c693c44abbb81577f\\
Author: lijian $<lijian@os-class.engr.oregonstate.edu>$\\
Date:   Thu Oct 22 19:25:25 2015 -0700\\
con2 finished\\


\hline
commit cbe43c546bf312ec1ba0831c187c3d02848984e3\\
Author: lijian $<lijian@os-class.engr.oregonstate.edu>$\\
Date:   Mon Oct 19 20:39:18 2015 -0700\\
project 2 created\\


\hline
commit ccd3d15153e9973c14c01faa0dcdb87175764ed8\\
Author: lijian $<lijian@os-class.engr.oregonstate.edu>$\\
Date:   Mon Oct 19 12:41:44 2015 -0700\\
start working with con 2\\

\hline

 
\end{tabular}
\end{table}








\end{document}