\documentclass[letterpaper,11pt,titlepage]{article}

\usepackage{graphicx}                                        

\usepackage{amssymb}                                         
\usepackage{amsmath}                                         
\usepackage{amsthm}                                          

\usepackage{alltt}                                           
\usepackage{float}
\usepackage{color}

\usepackage{url}

\usepackage{balance}
\usepackage[TABBOTCAP, tight]{subfigure}
\usepackage{enumitem}

\usepackage{pstricks, pst-node}

\usepackage{geometry}
\geometry{textheight=10in, textwidth=7.5in}

\usepackage{listings}

\definecolor{mygreen}{rgb}{0,0.6,0}
\definecolor{mygray}{rgb}{0.5,0.5,0.5}
\definecolor{mymauve}{rgb}{0.58,0,0.82}

\lstset{ %
  basicstyle=\ttfamily,            % the size of the fonts that are used for the code
  breakatwhitespace=false,         % sets if automatic breaks should only happen at whitespace
  breaklines=true,                 % sets automatic line breaking
  captionpos=b,                    % sets the caption-position to bottom
  commentstyle=\color{mygreen},    % comment style
  escapeinside={\%*}{*)},          % if you want to add LaTeX within your code
  extendedchars=true,              % lets you use non-ASCII characters; for 8-bits encodings only, does not work with UTF-8
  keepspaces=true,                 % keeps spaces in text, useful for keeping indentation of code (possibly needs columns=flexible)
  keywordstyle=\color{blue},       % keyword style
  numbers=left,                    % where to put the line-numbers; possible values are (none, left, right)
  numbersep=10pt,                  % how far the line-numbers are from the code
  numberstyle=\tiny\color{mygray}, % the style that is used for the line-numbers
  rulecolor=\color{black},         % if not set, the frame-color may be changed on line-breaks within not-black text (e.g. comments (green here))
  showspaces=false,                % show spaces everywhere adding particular underscores; it overrides 'showstringspaces'
  showstringspaces=false,          % underline spaces within strings only
  showtabs=false,                  % show tabs within strings adding particular underscores
  stepnumber=1,                    % the step between two line-numbers. If it's 1, each line will be numbered
  stringstyle=\color{mymauve},     % string literal style
  tabsize=8,                       % sets default tabsize to 8 spaces
  %title=\lstname                  % show the filename of files included with \lstinputlisting; also try caption instead of title
}

\lstdefinestyle{niceC}{
  belowcaptionskip=1\baselineskip,
  breaklines=true,
  frame=L,
  xleftmargin=0.2in,
  language=C,
  showstringspaces=false,
  basicstyle=\footnotesize\ttfamily,
  keywordstyle=\bfseries\color{green!40!black},
  commentstyle=\itshape\color{purple!40!black},
  identifierstyle=\color{blue},
  stringstyle=\color{orange},
}

%random comment

\newcommand{\cred}[1]{{\color{red}#1}}
\newcommand{\cblue}[1]{{\color{blue}#1}}

\usepackage{hyperref}

\def\name{Jianzhi Li}

%% The following metadata will show up in the PDF properties
\hypersetup{
  colorlinks = true,
  urlcolor = black,
  pdfauthor = {\name},
  pdfkeywords = {cs444 ``operating systems II'' Project 1},
  pdftitle = {CS 444 Project 1: Write-up},
  pdfsubject = {CS 444 Project 1},
  pdfpagemode = UseNone
}

\parindent = 0.0 in
\parskip = 0.2 in

\usepackage{sectsty}

\sectionfont{\fontsize{12}{15}\selectfont}


\begin{document}


Jianzhi Li\newline
CS444 Project 1 Write-up

\section{Commands Used To Perform The Requested Actions}


\subsection{Repository Creation}

\begin{lstlisting}
git clone git.github.com/leejianzhi/cs444.git
git config --global user.name "leejianzhi"
git config --global user.email "lijian@oregonstate.edu"
git init
git remote add origin git.github.com/leejianzhi/cs444.git
git push origin master
\end{lstlisting}


\subsection{Run QEMU}

\begin{lstlisting}
git clone git://git.yoctoproject.org/linux-yocto-3.14
source /scratch/opt/environment-setup-i586-poky-linux.csh
qemu-system-i386 -gdb tcp::5591 -S -nographic -kernel bzImage-qemux86.bin -drive file=core-image-lsb-sdk-qemux86.ext3,if=virtio -enable-kvm -net none -usb -localtime --no-reboot --append "root=/dev/vda rw console=ttyS0 debug".
gdb
target remote :5591
cp /scratch/spring2015/files/config-3.14.26-yocto-qemu /scratch/fall2015/cs444-91/linux-yocto-3.14/.config
make -j4 all
\end{lstlisting}

\section{Concurrency Solution}

\begin{enumerate}



\item 
\end{enumerate}

\section{Questions}
\begin{enumerate}

\item For concurrency exercise that reminds me some main knowledge related to CS344 such as how Interprocess Communications and synchronous worked in operating system.
Otherwise, this exercise also related to assembly language that is related what I have studied in CS271. But since that is been a long time with that class, those are 
almost new knowledge for me and that is a nice warm up with assembly language for next few assignments. However, there has another point for this exercise is to thinking
about parallel programming that is also why we have producer and consumer to process two queues of data. 

\item For approach this problem, from the introduction of this exercise, I could know my design is have to have two independently threads handler which are producer and consumer.
Each of those handler should handle producer threads and consumer threads separately which mean those threads are parallel. To be parallel that means threads working for producer that they
are only allowed process by producer, threads for consumers also only can process by consumers. Two queues of threads they do not have competition because they have privileges handler which are 
producer and consumer. To match the requirement, I would have to use multiple threads and mutex lock since the buffer only can hold 32 items. Also I should handle conditions, if consumer is want to 
have an item from buffer but the buffer is empty I should let consumer to be wait until the process gives an item to buffer. Otherwise, producer should wait when the buffer is full. In this situation, timing design 
is also important. The hardest part for design this program is to user assembly language to handle random values to distribute to producer. For end of the design, I still do not have any idea to handle that by using 
Mersenne Twister to handle it on os-class. For algorithm, for this project there is do not have any deeply algorithm to design and use. The only one is should be generate random values for the producer but I have not
figure it out.

\item For testing this program, I want to print out the random values generated from the assembly code, either the number and the waiting period. Also print out what value the producer send to buffer and what consumer received from the buffer. 
Also I should test when buffer is empty, does consumer can wait until there is something in the buffer. For testing producer, I would comment the consumer function that makes producer function never stop push items to the buffer
until that is full. If the producer can blocked until there has a space in the buffer.

\item I both reviewed the knowledge from CS344 and learn how to use mix of C programming language and assembly language. For using assembly language in this class is important since I feel like we study more about kernel we are more close to the hardware which requires me 
to know how to use assembly language to approach the bottom level programming.

\end{enumerate} 




\section{Verison Control Log}


\section{Word Log}



\section{Questions}



\end{document}